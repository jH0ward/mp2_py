\documentclass[12pt]{article}
\usepackage{amsmath}
\usepackage{setspace}
\usepackage{fullpage}
\begin{document}
\begin{center}
{\huge Polyatomic Self-Consistent Field \\ with Perturbative Energy Correction: \\ 
\vspace{0.2cm}Dynamical Electron Correlation} \\
\vspace{2in}
{\large Coleman Howard \\
May 9, 2012 \\
\vspace{1cm}
}
\end{center}
\thispagestyle{empty} \phantom{.}
\newpage
\setcounter{page}{1}
\section{Introduction}
In my last project, I demonstrated the Hartree-Fock (HF) approximation to the time-independent
Schr\"{o}dinger equation. HF theory is a starting point for many of the most successful
\emph {ab initio} methods commonly used in quantum chemistry research today.
More accurate methods employ the HF wave function as a zeroth or first order
approximation because the electronic energy computed within the approximation
has less than 1\% error. The major source of error in HF is due to the fact that electron
repulsion is treated in an average sense. A truly accurate description of electrons would
account for the correlated motions of the electrons in a dynamical sense, as
electrons experience instantaneous mutual repulsion. Here, I will demonstrate
how perturbation theory is used to recover some of this dynamical correlation.
Also, this work will improve on the previous effort by considering atoms larger than 
hydrogen and helium, which only possess spherical electron probability distributions.
Charge distributions in larger atoms are not strictly spherical, significantly 
complicating the calculation of the required molecular integrals.

The goal is to calculate the total energy of a single water molecule (H$_2$O).
To recover some of the dynamical electron correlation, we will compute the second-order
M{\o}ller-Plesset energy (MP2 energy.) Later, we will examine the form of the final expression
to compute this energy without a lengthy derivation. For now, we will simply say that
the MP2 energy is computed from the HF energy with a perturbative correction that 
includes contributions from the electron-repulsion integrals over unoccupied orbitals.
The HF wave function is calculated in a self-consistent field (SCF) procedure, in
the exact same manner as the previous work. However, the integrals required now involve
six-dimensional integrals over basis functions representing non-spherical electron
probability distributions. Let's begin by examining the mathematical form of the basis functions
and the kinds of terms that result when trying to compute the four different types of
integrals required in the SCF procedure.

The basis function utilized here is the STO-3G basis set, referred to as a minimal basis
set because it contains only one basis function for each occupied orbital for each atom. 
For example, the hydrogen atom has one electron, which is said to reside in an \emph{s}
orbital, where the probability distribution is spherically symmetric. The helium atom has
two electrons which are said to reside in an {\emph s} orbital. Oxygen, with eight electrons,
has two electrons occupying an {\emph s} orbital (1s), two more electrons occupying a different 
\emph{s} orbital (2s), and four electrons distributed among what are referred to as \emph{p}
orbitals (2p). The labels \emph{s} and \emph{p} actually come from spectroscopic
terminology in reference to characteristic lines from atomic spectra: \emph{s} for
sharp and \emph{p} for principal. The 3-dimensional pictorial representation of a \emph{p}
orbital is commonly shown in undergraduate chemistry textbooks as having a
dumbbell shape.

The STO-3G basis set for the hydrogen atom takes the following form:
\begin{align*}
\phi_{1s} &= \displaystyle\sum\limits_{i=1}^3 d_{i,1s} g_{1s}(\alpha_{i,1s}) \\
g_{1s}(\alpha_{i,1s}) &= N \mathrm{exp}[-\alpha_i r^2]
\end{align*}
In the first expression, the contraction coefficients $d_i$ and exponents
$\alpha_i$ are predefined, and have been
chosen to closely approximate the canonical wave function solutions
to the hydrogen atom Schr\"{o}dinger equation. The second expression shows
the form of a normalized primitive Gaussian, which when combined in the 
contracted series of the first expression, defines a single basis function.
Similarly, the STO-3G basis set for the oxygen atom is defined below:
\begin{align*}
\phi_{1s} &= \displaystyle\sum\limits_{i=1}^3 d_{i,1s} g_{1s}(\alpha_{i,1s}) \\
\phi_{2s} &= \displaystyle\sum\limits_{i=1}^3 d_{i,2s} g_{1s}(\alpha_{i,2sp}) \\
\phi_{2p_x} &= \displaystyle\sum\limits_{i=1}^3 d_{i,2p_x} g_{2p}(\alpha_{i,2sp}) \\
\phi_{2p_y} &= \displaystyle\sum\limits_{i=1}^3 d_{i,2p_y} g_{2p}(\alpha_{i,2sp}) \\
\phi_{2p_z} &= \displaystyle\sum\limits_{i=1}^3 d_{i,2p_z} g_{2p}(\alpha_{i,2sp}) \\
g_{2p_x}(\alpha_{i,1s}) &= N \ x  \ \mathrm{exp}[-\alpha_i r^2]
\end{align*}
The subscripts above are meaningful and indicate that the contraction coefficients
are the same for all 1s functions, regardless of the atom type. Notice also that
the gaussian form of the 1s function will be the same as the 2s function, with only
the exponents changing. The last expression above shows the Gaussian functional
form for the 2p$_x$ function. The 2p$_y$ and 2p$_z$ functions take analogous forms.
So with the 5 basis functions for the oxygen atom and 1 basis function for each of the hydrogen
atoms, we are ready to look at the molecular integrals over these functions.
\section{Molecular Integral Evaluation}
There are four types of molecular integrals that need to be computed: overlap integrals,
kinetic energy integrals, nuclear attraction integrals, and electron-repulsion integrals.
The first three categories are one-electron integrals over the x,y and z Cartesian components
of space. Each of these types of integrals consist of a basis function on the left-hand side of
the integral, an operator in the middle, and another basis function on the right-hand side.
We defined a basis function earlier as being a linear combination of contracted primitive Gaussian
functions. So each integral involving two basis functions can be broken down into a sum
of integrals over the two basis functions' component primitive Gaussian functions.
The problem is more straightforward when cast in this manner, and it allows us to ignore the
corresponding contraction coefficients while we evaluate the integral over two primitive
Gaussians.

\subsection{Overlap Integrals}
The first class of integrals we need to evaluate is the overlap integral. Here, the operator
in between the two basis functions is simply unity. The overlap integral between
Gaussian primitives 1 and 2 ($S_{12}$) is given by the expression below:
\begin{align*}
S_{12} &= \int g_1(\alpha_1,{\bf A},l_1,m_1,n_1) \ g_2(\alpha_2,{\bf B},l_2,m_2,n_2)\,d{\bf r}
\end{align*}
Here, {\bf A} and {\bf B} represent the 3-dimensional spatial components of the two
primitives' centers (i.e., the centers of the atoms which the functions represent.)
Application of the Gaussian Product Theorem leads to a manageable expression for
the overlap:
\begin{align*}
S_{12}&=\mathrm{exp}[-\alpha_1\alpha_2(\overline{AB})^2/\gamma]I_xI_yI_z \\
\gamma &= \alpha_1+\alpha_2 \\
I_x &=\displaystyle\sum\limits_{i=0}^{l_1+l_2}f_i(l_1,l_2,\overline{PA}_x,\overline{PB}_x)
\int^\infty_{-\infty} x_P^i \mathrm{exp}[-\gamma x_P^2] \\
&= \displaystyle\sum\limits_{i=0}^{l_1+l_2}f_i(l_1,l_2,\overline{PA}_x,\overline{PB}_x)
\frac{ (2i-1)!! } { (2\gamma)^i} \left( \frac{\pi}{\gamma}\right) ^{1/2} \\
f_k &= \displaystyle\sum\limits_{q=\mathrm{max}(-k,k-2l_2)}^{\mathrm{min}(k,2l_1-k)}
{l_1 \choose i} {l_2 \choose j} (\overline {\bf PA})_x^{l_1-i} ({\bf PB})_x^{l_2-j}
\end{align*}
In the preceding expressions and in future ones, $l_1$ and $l_2$ refer to angular momentum
vector components in the x-direction associated with a particular primitive Gaussian. For
our purposes, $l$ will equal 1 for a $p_x$ function and 0 in all other cases. Likewise, the letters
$m$ and $n$ serve the same purpose for angular momentum in the y and z directions,
respectively. The letter {\bf P} will be reserved for the center of a new Gaussian function
formed from the two primitive Gaussians by using the Gaussian Product theorem.

From a programming standpoint, the overlap integral over two primitive Gaussians is not
particularly daunting. The expression above is actually general for Gaussian basis 
functions of arbitrary angular momentum, not just $s$ and $p$ functions. The only piece
of the expressions above that I have not coded before is the double factorial. However,
the code is just as simple as the single factorial, except a multiplier value is decremented by
2 each time, instead of 1. The function I wrote to compute the overlap includes access
to factorial, double factorial, and binomial coefficient functions. 
The information required by the function is
all present in attributes associated with each basis function. So the main function sends
two basis set objects to the overlap function, along with two numbers specifying which 
primitives are to be used. These numbers allow the overlap function to retrieve information
such as $\alpha$ values and angular momentum components needed to perform
the calculation. The contraction coefficients are then applied to the result inside the main
function.

\subsection{Kinetic Energy Integrals}
The operator of interest for the kinetic energy integrals is the Cartesian Laplacian
operator.
Conveniently, the kinetic energy integral between 2 primitive Gaussians ($T_{12}$)
can be reduced to a series of overlap integrals:
\begin{align*}
T_{12}&=I_x + I_y + I_z \\
I_x &= \frac{1}{2}l_1l_2\langle-1\mid -1\rangle_x
      + 2\alpha_1\alpha_2\langle+1\mid+1\rangle_x \\
      & \ - \alpha_1 l_2 \langle +1\mid -1 \rangle_x - \alpha_2 l_2 \langle-1\mid+1\rangle_x \\
 \langle +1\mid_x &= x^{l+1}y^m z^n \mathrm{exp} [-\alpha r^2]
\end{align*}

So the kinetic energy integrals can be generated completely from a set of 
associated overlap integrals with angular momentum quantum numbers incremented
or decremented by one unit.

\subsection{Nuclear-Attraction Integrals}
The operator inside the nuclear-attraction integrals is ${\bf R}_C^{-1}$,
the displacement vector from a basis function to a nuclear center.
Unfortunately, with the introduction of the nuclear-attraction integral, 
we have encountered an integral which is not separable due to the nature of
the operator. A transformation is required, and there are multiple
equivalent techniques. The method I have chosen for this work is using
recurrence relations. This was chosen simply because it looked like the easiest to
code.  It is based on the method of Obara and Saika, who showed
how to arrive at the solution to the Cartesian integral without having
to compute intermediate Hermite integrals as in other methods.
The method is best described by the concept of a source integral
and a target integral:
\begin{align*}
\Theta_{000000}^N &=\frac{2\pi}{\gamma}K_{ab}^{xyz}F_N(\alpha R_{PC}^2) \\
\Theta_{ijklmn}^0 &= V_{ijklmn}^{000}
\end{align*}
The first expression is the source integral, meaning the starting point.
I can set up a function to compute this expression. The second expression, the target integral,
refers to the particular integral of interest. The goal is to use recursion relations
to express the target integral in terms of the known source integral.
In the source integral, the first two indices $i$ and $j$ refer to the values of $l_1$ and
$l_2$ (i.e., the angular momentum quantum numbers of the primitive Gaussian
functions in the x-direction.) The remaining four lower indices refer to the two Gaussians'
y- and z-components of quantized angular momentum.
We will now define the crucial recursion relation:
\begin{align*}
\Theta_{i+1,j,k,l,m,n}^N &= X_{PA}\Theta_{ijklmn}^N + \frac{1}{2\gamma}
\left(i\Theta_{i-1,j,k,l,m,n}^N+j\Theta_{i,j-1,k,l,m,n}^N\right) \\
& \ -X_{PC}\Theta_{ijklmn}^{N+1} - \frac{1}{2\gamma}
\left(i\Theta_{i-1,j,k,l,m,n}^{N+1} + j\Theta_{i,j-1,k,l,m,n}^{N+1}\right)
\end{align*}
Analogous expressions are obtained for incrementing or decrementing
the other indices. But let us look at the evaluation of the source integral
in some detail. In the source integral, the key to evaluation is the
approximation to the Boys function ($F_N$). The preceding factors are easily
calculated from the defined qualities of the basis functions. The Boys
function of order $N$ is defined in the following manner:
\begin{align*}
F_N(x) &= \int^1_0 \mathrm{exp} [-xt^2] t^{2N}\ dx \\
F_N(0) &= \frac{1}{2n+1} \\
F_N(x) &\approx \frac{(2n-1)!!}{2^{n+1}}\sqrt{\frac{\pi}{x^{2n+1}}} \ \ \mathrm{  (large \ x)}
\end{align*}
The Boys functions is actually the only numerical approximation that must be made
in this  program. It is also used in the computation of the electron-repulsion
integrals so an accurate approximation is essential. The strategy used in this
program is that suggested by Helgaker, et al. The domain of the Boys function
is divided into 3 regions:  (i) x $<$0.18, (ii) 0.18 $<$ x $<$ 19.35 and (iii) x $>$ 19.35.
For the first region, a sixth-order Taylor series expansion is stable, resulting in greater than 10
decimal points of precision. Let's examine the Python code for this series: \\
\\
\hspace{0.15in}{\tt def TAYLOR(x,order,sub): }

\hspace{0.5in}{\tt total=0.0}

\hspace{0.5in}{\tt for k in range(order+1):}

\hspace{0.85in}{\tt total+=((-1*x)**k)/factorial(k)/(2*sub+2*k+1)}

\hspace{0.5in}{\tt return total} \\
\\
Here, the variable $sub$ refers to the order of the Boys function, while
$order$ refers to the order of the Taylor series expansion, which is chosen
as 6 in all cases. In the third region of the Boys domain (large x), the
third expression above can be used. Now, the most difficult region is
the middle region where neither a Taylor expansion nor the large x
expression is stable. The solution recommended by Helgaker is to 
tabulate the Boys function values from 0.18 to 18.35 in increments of 0.1,
and so the approximation in this region can be found by expanding around
the nearest tabulated point. This strategy leads to convergence with errors
smaller than $10^{-14}$. The values were tabulated using Mathematica software
for Boys function orders 1 through 8. These values were placed into
8 separate dictionaries in one file, mapping an $x$ value to the corresponding
Mathematica result. Here is the function:
\\

\hspace{0.15in}{\tt def BOYS(x,N,order=6): }

\hspace{0.5in}{\tt total=0.0}

\hspace{0.5in}{\tt xten=x*10}

\hspace{0.5in}{\tt closest=round(xten)/10}

\hspace{0.5in}{\tt diff=x-closest}

\hspace{0.5in}{\tt for k in range(order+1):}

\hspace{0.85in}{\tt marker=k+N}

\hspace{0.85in}{\tt if marker==1:}

\hspace{1.0in}{\tt mydict=boysONE}

\hspace{0.85in}{\tt if marker==2:}

\hspace{1.0in}{\tt mydict=boysTWO}

\hspace{0.85in}{\tt if marker==3:}

\hspace{1.0in}{\tt mydict=boysTHREE}

\hspace{0.85in}{\tt if marker==4:}

\hspace{1.0in}{\tt mydict=boysFOUR}

\hspace{0.85in}{\tt if marker==5:}

\hspace{1.0in}{\tt mydict=boysFIVE}

\hspace{0.85in}{\tt if marker==6:}

\hspace{1.0in}{\tt mydict=boysSIX}

\hspace{0.85in}{\tt if marker==7:}

\hspace{1.0in}{\tt mydict=boysSEVEN}

\hspace{0.85in}{\tt if marker==8:}

\hspace{1.0in}{\tt mydict=boysEIGHT}

\hspace{0.85in}{\tt total+=mydict[closest]*(-1*diff)**k/factorial(k)}

\hspace{0.5in}{\tt return total}\\

\noindent
So with a stable approximation to Boys function, computing the nuclear attraction
integrals is as simple as deriving the target integral in terms of the Boys
function source integral. At this point, I would like to explain my ingenious
algorithm for deriving the recursion relations, but I don't have one.
And I doubt it would enhance the program if I did. Because this computation
is only dealing with a maximum angular momentum quantum of $1$,
there is no clear advantage to coding a smart algorithm rather than deriving
the few needed equations by hand. In addition, to extend this program to the $d$
or $f$ functions of angular momenta 2 and 3, respectively, hard coding would still
probably be the preferred method. In my research, I have never used a function
with angular momenta higher than 3.

\subsection{Electron Repulsion Integrals}
The last set of integrals is defined by the operator $\bf{r}_{12}^{-1}$, representing
the distance between electrons.
For the electron-repulsion integrals (ERI), the recurrence relations of Obara and Saika
are called upon once again. The source and target integrals are defined below:
\begin{align*}
\Theta_{0000;0000;0000}^N &=\frac{2\pi^{5/2}}{pq\sqrt{p+q}}K_{ab}^{xyz}K_{cd}^{xyz}
F_N(\alpha R_{PQ}^2) \\
\Theta_{i_x j_x k_x; j_y k_y l_y; j_z k_z l_z}^0 &= g_{i_x j_x k_x; j_y k_y l_y; j_z k_z l_z}
\end{align*}
So there are 12 lower indices in all, 3 angular momentum quantum numbers for each of the
four primitive Gaussians. Here, $p$ has replaced $\gamma$ and $q$ is the
corresponding value for the Gaussian product of the two primitives on the
right hand side of the integral. Luckily the recurrence relations for the
x-components are not affected by those of the y- or z-components, and vice versa.
Let's examine the form of the Obara-Saika two-electron recurrence relation in the
x direction. We will drop the unaffected indices.

\begin{align*}
\Theta_{i+1,j,k,l}^N &= X_{PA}\Theta_{ijkl}^N - \frac{\alpha}{p}X_{PQ}
\Theta_{ijkl}^{N+1} + \frac{i}{2p}\left( \Theta_{i-1,j,k,l}^N - \frac{\alpha}{p}
\Theta_{i-1,j,k,l}^{N+1} \right) \\
&+ \frac{j}{2p} \left( \Theta_{i,j-1,k,l}^N - \frac {\alpha}{p}\Theta_{i,j-1,k,l}^{N+1} \right)
+ \frac{k}{2(p+q)}\Theta_{i,j,k-1,l}^{N+1} + \frac{l}{2(p+q)}\Theta_{i,j,k,l-1}^{N+1}
\end{align*}
\noindent
Again, the programmer is left with some equations to derive, but all of the information needed
is conveniently present inside the basis function object from the class {\tt Gaussian3D}.
And the concept needed to compute the source integral (lower indices all
equal to zero) has already been used to compute the nuclear attraction integrals.

\section{Perturbative Energy Correction}
MP2 has been recognized as an efficient and relatively simple method of recovering 
dynamical electron correlation missing from the HF energy. The method uses
the so-called ``canonical orbitals'' from the SCF procedure. The procedure for
obtaining these has been explained in the past effort. Now to obtain the MP2
energy only two steps are required: (i) a transformation of the atomic orbital (AO)
basis into a molecular orbital (MO) basis and (ii) the calculation of the energy.
The basis transformation need only be applied to the matrix containing the
results of the ERI. A loop similar to the following accomplishes our goal:

\hspace{0.15in}{\tt for p in xrange(nbasis):

\hspace{0.3in}for q in xrange(nbasis):

\hspace{0.45in}for r in xrange(nbasis):

\hspace{0.6in}for s in xrange(nbasis):

\hspace{0.75in}val=0.0

\hspace{0.75in}for i in xrange(len(C)):

\hspace{0.9in}for j in xrange(len(C)):

\hspace{1.05in}for k in xrange(len(C)):

\hspace{1.2in}for l in xrange(len(C)):

\hspace{1.35in}val+=C[i][p]*C[j][q]*ERI[i][j][k][l]*C[k][r]*C[l][s]

\hspace{0.75in}newERI[p][q][r][s]+=val}
\\

\noindent
This 8-layer loop is responsible for the larger scaling of the MP2 method.
The above algorithm is the naive method, as smarter loops can
scale as $N^5$, but this exercise is not for achieving efficiency.
After transforming the ERI into the MO basis ({\tt newERI}),
we have everything needed to compute the MP2 energy:
\\

{\tt
\hspace{0.15in}emp2=0.0

\hspace{0.15in}for i in range(ndocc):

\hspace{0.3in}for a in range(ndocc,nbasis):

\hspace{0.45in}for j in range(ndocc):

\hspace{0.6in}for b in range(ndocc,nbasis):

\hspace{0.75in}emp2+=newERI[i][a][j][b]*(2*newERI[i][a][j][b]-newERI[i][b][j][a])

\hspace{1.1in} / (E[i][i]+E[j][j]-E[a][a]-E[b][b])
}
\\

\noindent
So emp2 represents the dynamical electron correlation missing from the HF energy
computation. In the above expression, the {\tt E} values referenced are those
from the SCF procedure obtained by diagonalizing the Fock matrix. {\tt ndocc}
is the number of doubly-occupied orbitals in the molecular system. If we
assume all electrons are paired, then {\tt ndocc} equals half the number of 
electrons.
\section{Conclusions}
The main result of the program is that it works. The integrals computed match
those of {\tt Gaussian09} to as many decimal places as {\tt Gaussian09} will print.
Also the MP2 energy matches to as many decimal places as I want. The number
of decimal places that can match on the MP2 energy is limited to the convergence
of the Hartree-Fock wave function. For this program, I have set an extremely tight
SCF convergence criterion of a root mean square (RMS) deviation in the density matrix
between successive iterations of less than $1\times10^{-17}$.
Default convergence in {\tt Gaussian09} is $1\times 10^{-8}$. So this program's
convergence is very tight. The greatest difficulty in this program was computing
the electron-repulsion integrals. The code is both unattractive and inefficient, but
it does produce accurate results. Currently, the best algorithms for computing these
integrals involve starting with a tree-search routine for finding the most efficient way to
only compute the necessary integrals. A result of using recurrence relations to
compute integrals is that the repulsion integrals associated with lower-order
angular momentum quantum numbers must be computed to obtain the higher-order
results. On one hand, this could be advantageous if the results were stored and
recalled efficiently since they will be used often. On the other hand, this means
that recurrence relations could require some extra floating point operations which may
exceed the demand of other computational schemes.  

\section{References}
All of the information here is from \emph{Molecular Electronic-Structure
Theory}, T. Helgaker, P. J J{\o}rgensen, and J. Olsen, Wiley, {\bf 2000}.



\end{document}




